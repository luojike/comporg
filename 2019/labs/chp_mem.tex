
%------------------------------------------------------------------------------
\chapter{存储器功能模拟程序设计}

%------------------------------------------------------------------------------
\section{实验目标}

\begin{enumerate}
	\item 学习掌握存储器的功能和操作;
	\item 学习了解模拟程序设计方法。
\end{enumerate}

%------------------------------------------------------------------------------
\section{实验条件}

\begin{enumerate}
	\item Linux操作系统(推荐),或Windows操作系统;
	\item 程序开发工具GCC(Linux),或DevC++(Windows)。
\end{enumerate}

%------------------------------------------------------------------------------
\section{实验准备}

\begin{enumerate}
	\item 回顾教材第五章关于存储器的内容,了解不同数据类型在存储器中存放规则及其读、写操作;
	\item 利用图书馆或互联网资源, 了解模拟程序设计(例如: https://fms.komkon.
		org/EMUL8/HOWTO.html )。
\end{enumerate}

%------------------------------------------------------------------------------
\section{实验内容}

\begin{enumerate}
	\item 从实验说明文档包中找到 mem.cpp 程序,了解其中思路,补全write操作;
		%\lstinputlisting[language=c++]{stacktest.cpp}
	\item 编译运行修改好的 mem.cpp 程序,观察程序运行结果。如果运行结果有错误,请改正。
\end{enumerate}

%------------------------------------------------------------------------------
\section{思考问题}

\begin{enumerate}
	\item 存储器的最基本存储单位是什么?存储器大小一般以什么为单位?
	\item C/C++语言中各种基本类型(char, int, float, double)的数据在存储器中是如何存放的?
	\item C/C++语言中的指针是不是完全等同于存储器的地址?
\end{enumerate}

%------------------------------------------------------------------------------
\section{报告要求}

\begin{enumerate}
	\item 实验报告需要有姓名,学号,班级,实验名称,实验目标,实验条件,实验内容,实验记录,实验分析等项目;
	\item 实验记录需要有如下信息:(1)实验机器CPU配置,内存配置,操作系统名称和版本号,GCC或DevC++版本号;(2)补全write操作后 mem.cpp 程序源代码,以及GCC编译命令文本;(3)模拟程序运行结果(如果结果难以截图,请文字描述)。
	\item 实验分析要根据实验记录得到的结果,回答思考问题中各个问题。
\end{enumerate}

