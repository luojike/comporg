
%------------------------------------------------------------------------------
\chapter{整数的表示}

%------------------------------------------------------------------------------
\section{实验目标}

\begin{enumerate}
	\item 学习使用定长整数类型;
	\item 学习使用位操作对数据位进行操作。
\end{enumerate}

%------------------------------------------------------------------------------
\section{实验条件}

\begin{enumerate}
	\item Linux操作系统(推荐),或Windows操作系统;
	\item 程序开发工具GCC(Linux),或DevC++(Windows)。
\end{enumerate}

%------------------------------------------------------------------------------
\section{实验准备}

\begin{enumerate}
	\item 利用图书馆或互联网资源,了解标准C/C++头文件stdint.h或cstdint中定义的int8\_t, int16\_t, int32\_t, int64\_t, uint8\_t, uint16\_t, uint32\_t, uint64\_t等数据类型,以及(U)INT8(16,32,64)\_MIN(MAX)等预定义常量(例如: http://www.cplusplus.com/reference/cstdint/ )。
	\item 利用图书馆或互联网资源,了解位操作用法(例如: https://en.wikipedia.
		org/wiki/Bitwise\_operations\_in\_C )。
\end{enumerate}

%------------------------------------------------------------------------------
\section{实验内容}

\begin{enumerate}
	\item 编写一个程序,利用位操作检查一个整型数据的各个数据位并输出,或者说,输出一个整型数的二进制表示;
	\item 用8位,16位,32位,64位的无符号和有符号整型数据0, $\pm$1(无符号数忽略~-1), 最大值, 最小值测试上述程序,检验结果的正确性。
\end{enumerate}

%------------------------------------------------------------------------------
\section{思考问题}

\begin{enumerate}
	\item C/C++语言中无符号整型数的编码是什么?有符号整型数的编码是什么?
	\item C/C++语言中按位与,按位或,按位非,按位异或对有符号整型数的符号位是如何处理的?
	\item C/C++语言中移位操作对有符号整型数的符号位是如何处理的?
\end{enumerate}

%------------------------------------------------------------------------------
\section{报告要求}

\begin{enumerate}
	\item 实验报告需要有姓名,学号,班级,实验名称,实验目标,实验条件,实验内容,实验记录,实验分析等项目;
	\item 实验记录需要有如下信息:(1)实验机器CPU配置,内存配置,操作系统名称和版本号,GCC或DevC++版本号;(2)整型数二进制输出程序源代码,GCC编译命令文本;(3)用8位,16位,32位,64位的无符号和有符号整型数据0, $\pm$1(无符号数忽略~-1), 最大值, 最小值测试上述程序,将输出的二进制与手工计算结果相比较,检验结果的正确性。
	\item 实验分析要根据实验记录得到的结果,回答思考问题中各个问题。
\end{enumerate}

