
%------------------------------------------------------------------------------
\chapter{栈与过程调用}

本实验中使用的 stacktest.cpp 程序是受到通信1603班崔威恒同学和通信1604班庄义昱同学在讨论课上所演示程序的启发而编写,特此鸣谢!

%------------------------------------------------------------------------------
\section{实验目标}

\begin{enumerate}
	\item 学习掌握栈在过程调用中的作用;
	\item 学习了解栈溢出的原理。
\end{enumerate}

%------------------------------------------------------------------------------
\section{实验条件}

\begin{enumerate}
	\item Linux操作系统(推荐),或Windows操作系统;
	\item 程序开发工具GCC(Linux),或DevC++(Windows)。
\end{enumerate}

%------------------------------------------------------------------------------
\section{实验准备}

\begin{enumerate}
	\item 回顾教材第二章关于过程调用的内容,了解栈在过程调用中保存返回地址,传递参数以及为局部变量提供存储空间等作用;
	\item 利用图书馆或互联网资源,了解栈溢出的原因及影响(例如: https://en.
		wikipedia.org/wiki/Stack\_buffer\_overflow )。
\end{enumerate}

%------------------------------------------------------------------------------
\section{实验内容}

\begin{enumerate}
	\item 从实验说明文档包中找到 stacktest.cpp 程序,编译并运行;
		%\lstinputlisting[language=c++]{stacktest.cpp}
	\item 测试程序输入数从0到5的执行情况并截图记录, 以便后续分析。
\end{enumerate}

%------------------------------------------------------------------------------
\section{思考问题}

\begin{enumerate}
	\item C/C++语言的过程调用在利用栈传递参数时各个参数的入栈次序是怎么样的?反映到参数地址上的表现是什么?
	\item C/C++语言的过程调用在利用栈保存返回地址时,返回地址的保存位置与参数和局部变量保存位置的关系是怎么样的?
	\item 利用数组越界修改返回地址时可能有哪些情况? 修改成功时程序执行情况会是什么样的?不成功时又会是什么样的?
\end{enumerate}

%------------------------------------------------------------------------------
\section{报告要求}

\begin{enumerate}
	\item 实验报告需要有姓名,学号,班级,实验名称,实验目标,实验条件,实验内容,实验记录,实验分析等项目;
	\item 实验记录需要有如下信息:(1)实验机器CPU配置,内存配置,操作系统名称和版本号,GCC或DevC++版本号;(2)程序源代码,以及GCC编译命令文本;(3)输入数从0到5时程序的输出(请截图)。
	\item 实验分析要根据实验记录得到的结果,回答思考问题中各个问题。
\end{enumerate}

