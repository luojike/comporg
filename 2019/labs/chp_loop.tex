
%------------------------------------------------------------------------------
\chapter{矩阵相乘循环次序对执行时间的影响}

%------------------------------------------------------------------------------
\section{实验目标}

\begin{enumerate}
	\item 了解高速缓存结构和工作原理;
	\item 学习了解访问次序对高速缓存命中率的影响。
\end{enumerate}

%------------------------------------------------------------------------------
\section{实验条件}

\begin{enumerate}
	\item Linux操作系统(推荐),或Windows操作系统;
	\item 程序开发工具GCC(Linux),或DevC++(Windows)。
\end{enumerate}

%------------------------------------------------------------------------------
\section{实验准备}

\begin{enumerate}
	\item 回顾教材第五章中关于高速缓存结构和工作原理的知识;
	\item 利用图书馆或互联网资源,了解C/C++数组的内存放置(例如: 

		http://c.biancheng.net/cpp/html/2920.html )。
\end{enumerate}

%------------------------------------------------------------------------------
\section{实验内容}

\begin{enumerate}
	\item 编写一个矩阵相乘(建议矩阵大小1000x1000)的程序,测试不同循环次序对执行时间的影响。其中循环次序如下所示:
\begin{lstlisting}[language={C++}]
// ijk
for(i=0; i<n; i++) {
    for(j=0; j<n; j++) {
        s = 0;
        for(k=0; k<n; k++) {
            s += a[i][k]*b[k][j];
        }
	c[i][j] = s;
    }
}

// ...

// jik
for(j=0; j<n; j++) {
    for(i=0; i<n; i++) {
        s = 0;
        for(k=0; k<n; k++) {
            s += a[i][k]*b[k][j];
        }
	c[i][j] = s;
    }
}

// ...

// kij
for(k=0; k<n; k++) {
    for(i=0; i<n; i++) {
        r = a[i][k];
        for(j=0; j<n; j++) {
            c[i][j] += r*b[k][j];
        }
    }
}

// ...

// ikj
for(i=0; i<n; i++) {
    for(k=0; k<n; k++) {
        r = a[i][k];
        for(j=0; j<n; j++) {
            c[i][j] += r*b[k][j];
        }
    }
}

// ...

// jki
for(j=0; j<n; j++) {
    for(k=0; k<n; k++) {
        r = b[k][j];
        for(i=0; i<n; i++) {
            c[i][j] += a[i][k]*r;
        }
    }
}

// ...

// kji
for(k=0; k<n; k++) {
    for(j=0; j<n; j++) {
        r = b[k][j];
        for(i=0; i<n; i++) {
            c[i][j] += a[i][k]*r;
        }
    }
}

// ...

\end{lstlisting}

	编译并改正可能的错误,使得程序运行正确;

\item 运行程序并记录不同次序循环执行时间(建议每种循环针对不同大小(n)的矩阵反复连续执行多次,取第二次以后的执行时间计算平均时间,从而减小“预热”时间的影响)。
\end{enumerate}

%------------------------------------------------------------------------------
\section{思考问题}

\begin{enumerate}
	\item 矩阵在内存中是怎么存放的?
	\item 不同循环次序计算矩阵相乘的内存访问顺序有什么区别?
	\item 不同循环次序导致执行时间差异的原因是什么?
\end{enumerate}

%------------------------------------------------------------------------------
\section{报告要求}

\begin{enumerate}
	\item 实验报告需要有姓名,学号,班级,实验名称,实验目标,实验条件,实验内容,实验记录,实验分析等项目;
	\item 实验记录需要有如下信息:(1)实验机器CPU配置,内存配置,操作系统名称和版本号,GCC或DevC++版本号;(2)程序源代码,GCC编译命令文本;(3)不同循环次序计算矩阵相乘的平均时间。
	\item 实验分析要根据实验记录得到的结果,回答思考问题中各个问题。
\end{enumerate}

