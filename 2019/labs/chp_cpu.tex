
%------------------------------------------------------------------------------
\chapter{CPU功能模拟程序设计}

%------------------------------------------------------------------------------
\section{实验目标}

\begin{enumerate}
	\item 学习掌握CPU的功能和指令执行过程;
	\item 学习了解模拟程序设计方法。
\end{enumerate}

%------------------------------------------------------------------------------
\section{实验条件}

\begin{enumerate}
	\item Linux操作系统(推荐),或Windows操作系统;
	\item 程序开发工具GCC(Linux),或DevC++(Windows)。
\end{enumerate}

%------------------------------------------------------------------------------
\section{实验准备}

\begin{enumerate}
	\item 回顾教材第四章关于CPU的内容,了解CPU执行指令的过程;
	\item 利用图书馆或互联网资源, 了解模拟程序设计(例如: https://fms.komkon.
		org/EMUL8/HOWTO.html )。
\end{enumerate}

%------------------------------------------------------------------------------
\section{实验内容}

\begin{enumerate}
	\item 从实验说明文档包中找到 cpu\_todo.cpp 程序,了解其中思路,补全其中欠缺部分;
		%\lstinputlisting[language=c++]{stacktest.cpp}
	\item 编译运行修改好的程序,观察程序运行结果。如果运行结果有错误,请改正。
\end{enumerate}

%------------------------------------------------------------------------------
\section{思考问题}

\begin{enumerate}
	\item CPU执行指令的基本步骤是哪几步?分别对应于模拟程序中哪些部分?
	\item 模拟程序采用了无限循环 while(1) 来表示CPU反复执行指令的过程,那么模拟程序退出循环的条件是什么?
	\item 程序计数器(PC)在模拟程序中是怎么更新的?请说明 beq 指令对 PC 更新的过程;
\end{enumerate}

%------------------------------------------------------------------------------
\section{报告要求}

\begin{enumerate}
	\item 实验报告需要有姓名,学号,班级,实验名称,实验目标,实验条件,实验内容,实验记录,实验分析等项目;
	\item 实验记录需要有如下信息:(1)实验机器CPU配置,内存配置,操作系统名称和版本号,GCC或DevC++版本号;(2)补全后模拟程序源代码,以及GCC编译命令文本;(3)模拟程序运行结果(如果结果难以截图,请文字描述)。
	\item 实验分析要根据实验记录得到的结果,回答思考问题中各个问题。
\end{enumerate}

